\section{Question 3: Experimental Results and Discussion}

\subsection{Scope and analysis method}
This section analyzes the measured spectral response (SR), external quantum efficiency (EQE), and reflectance for Regions R1, R2, and R3. The objective is to quantify photo-current generation and interpret why each region performs differently.

Two EQE datasets were analyzed:
\begin{itemize}
\item \textbf{Full-spectrum EQE} (300--1200 nm, no long-pass filter), representing response under broad illumination.
\item \textbf{Truncated EQE} (with 700 nm long-pass filter), representing the bottom silicon subcell response relevant to perovskite-silicon tandem operation.
\end{itemize}

The integrated short-circuit current density is obtained from EQE through
\[
J_{sc} = q \int EQE(\lambda)\,\Phi_{AM1.5G}(\lambda)\,d\lambda
\]
where $\Phi_{AM1.5G}(\lambda)$ is the AM1.5G photon flux. SR and EQE are linked by
\[
EQE(\lambda) = SR(\lambda)\,\frac{1240}{\lambda}
\]
with $\lambda$ in nm and SR in A/W. Therefore, any trend seen in SR should be physically consistent with the EQE trend.

\subsection{Integrated current results}
Table~\ref{tab:q3_jsc} summarizes the integrated current density extracted from the measured EQE files.

\begin{table}[h]
\centering
\caption{Integrated current density extracted from EQE data.}
\begin{tabular}{lccc}
\hline
Region & $J_{sc}$ from EQE (mA/cm$^2$) & Truncated $J_{sc}$ (700--1200 nm) (mA/cm$^2$) & Reduction (\%) \\
\hline
R1 & 37.7605 & 17.9600 & 52.44 \\
R2 & 36.0073 & 16.3192 & 54.68 \\
R3 & 34.6822 & 17.0999 & 50.70 \\
\hline
\end{tabular}
\label{tab:q3_jsc}
\end{table}

Key quantitative outcomes:
\begin{itemize}
\item Full-spectrum ranking: \textbf{R1 $>$ R2 $>$ R3}.
\item Truncated ranking (700+ nm): \textbf{R1 $>$ R3 $>$ R2}.
\item The long-pass condition reduces current by about 51--55\% for all three regions, consistent with removal of blue-visible photon contribution below 700 nm.
\end{itemize}

These results already indicate that Region R1 is the strongest performer in terms of total generated current, and that Region R2 loses more useful near-IR collection than Region R3 in the truncated case.

\subsection{EQE behavior and interpretation}
Figure~\ref{fig:q3_eqe} compares EQE for all regions with and without the 700 nm filter.

\begin{figure}[htbp]
\centering
\includegraphics[width=\linewidth]{figures_q3/q3_eqe_comparison.png}
\caption{Measured EQE spectra for R1, R2, and R3 without (left) and with (right) the 700 nm long-pass filter.}
\label{fig:q3_eqe}
\end{figure}

The EQE curves can be interpreted in three wavelength zones:
\begin{itemize}
\item \textbf{300--450 nm (short wavelength):} R3 is much lower than R1/R2 in the no-filter case, indicating stronger front-surface optical loss or shallower carrier collection effectiveness in that region.
\item \textbf{450--900 nm (plateau region):} All regions are high, but R1 stays consistently highest. This produces a larger area under the EQE curve and directly contributes to higher integrated $J_{sc}$.
\item \textbf{900--1100 nm (near band-edge region):} All curves decline, but the decline is steepest for R2 and R3. R1 maintains stronger response deeper into near-IR, indicating better long-wavelength collection.
\end{itemize}

At representative wavelengths:
\begin{itemize}
\item 700 nm EQE: R1 = 96.15\%, R2 = 93.88\%, R3 = 94.15\%.
\item 900 nm EQE: R1 = 87.09\%, R2 = 79.08\%, R3 = 83.20\%.
\item 1100 nm EQE: R1 = 36.11\%, R2 = 27.74\%, R3 = 26.87\%.
\end{itemize}

With long-pass filtering, sub-700 nm response is correctly suppressed and near-IR differences remain. This confirms that the R1 advantage is not only due to visible response; R1 is also stronger in the tandem-relevant near-IR regime.

\subsection{SR behavior and physical consistency}
Figure~\ref{fig:q3_sr} shows SR trends for the same conditions.

\begin{figure}[htbp]
\centering
\includegraphics[width=\linewidth]{figures_q3/q3_sr_comparison.png}
\caption{Measured SR spectra for R1, R2, and R3 without (left) and with (right) the 700 nm long-pass filter.}
\label{fig:q3_sr}
\end{figure}

The SR curves mirror the EQE observations:
\begin{itemize}
\item R1 provides the highest peak SR and stronger long-wavelength response.
\item R2 is lowest in the near-IR range where bottom-cell current is critical.
\item R3 is intermediate: generally lower than R1, but less degraded than R2 under truncated conditions.
\end{itemize}

Representative SR values:
\begin{itemize}
\item 700 nm SR: R1 = 0.543 A/W, R2 = 0.530 A/W, R3 = 0.532 A/W.
\item 900 nm SR: R1 = 0.632 A/W, R2 = 0.574 A/W, R3 = 0.604 A/W.
\item 1100 nm SR: R1 = 0.320 A/W, R2 = 0.246 A/W, R3 = 0.238 A/W.
\end{itemize}

Because SR and EQE are analytically linked, this agreement increases confidence that the measured trends are real device behavior rather than plotting artifacts.

\subsection{Reflectance-driven optical loss analysis}
Figure~\ref{fig:q3_reflectance} presents reflectance for all regions.

\begin{figure}[htbp]
\centering
\includegraphics[width=0.92\linewidth]{figures_q3/q3_reflectance_comparison.png}
\caption{Measured reflectance spectra for R1, R2, and R3.}
\label{fig:q3_reflectance}
\end{figure}

Mean reflectance over 300--1100 nm is:
\begin{itemize}
\item R1: 3.80\%
\item R2: 4.23\%
\item R3: 9.27\%
\end{itemize}

Important observations and their implications:
\begin{itemize}
\item \textbf{R3 has very high short-wavelength reflectance} (approximately 350--450 nm), which strongly reduces blue photon entry and explains R3's weak short-wavelength EQE.
\item \textbf{R3 reflectance rises more strongly near 900--1100 nm} than R1/R2, which is consistent with lower long-wavelength EQE near the band-edge.
\item \textbf{R1 maintains lower reflectance across most of the useful range}, supporting superior absorption and therefore higher generated current.
\end{itemize}

Selected reflectance values:
\begin{itemize}
\item 700 nm: R1 = 1.69\%, R2 = 1.98\%, R3 = 0.69\%.
\item 900 nm: R1 = 3.30\%, R2 = 3.69\%, R3 = 6.22\%.
\item 1100 nm: R1 = 20.50\%, R2 = 20.09\%, R3 = 26.09\%.
\end{itemize}

Although R3 is low at 700 nm, its broader high-reflectance behavior at short and very long wavelengths is more detrimental overall, which is why its integrated current remains below R1.

\subsection{Cross-comparison and engineering interpretation}
To connect optical and electrical behavior:
\begin{itemize}
\item R1 combines low-to-moderate reflectance with the strongest EQE/SR across the middle and near-IR range, producing the largest integrated current.
\item R2 has reflectance close to R1 in many regions but weaker near-IR EQE/SR, suggesting additional collection losses (for example base-related or long-wavelength collection limits) rather than purely front optical loss.
\item R3 shows a mixed profile: optical penalties in blue and near band-edge are strong, but truncated near-IR current is still slightly better than R2 due to less severe 700--1000 nm degradation than R2.
\end{itemize}

For tandem relevance, the 700+ nm ranking matters most for the silicon bottom subcell. In this metric, R1 remains best, R3 is second, and R2 is last.

\subsection{Data quality, limitations, and uncertainty}
The analysis is based on 10 nm sampling from 300 to 1200 nm and direct instrument/software outputs for SR, EQE, and reflectance. The conclusions are robust at ranking level (R1 consistently best), but the following limitations should be acknowledged:
\begin{itemize}
\item 10 nm step size smooths fine spectral features.
\item Integration depends on calibration quality and reference detector stability.
\item Region-to-region comparison assumes consistent probe contact and measurement alignment.
\item The long-pass filter introduces an abrupt cutoff near 700 nm, so values close to cutoff should be interpreted with care.
\end{itemize}

These do not change the primary conclusion because the observed differences are systematic and repeated across multiple metrics (EQE, SR, and integrated $J_{sc}$).

\subsection{Conclusion for Question 3}
Question 3 requires quantitative integration of current and discussion of SR/reflectance behavior. Both are satisfied by the measured data. Region R1 delivers the highest full-spectrum and truncated integrated current density, with stronger near-IR EQE/SR and favorable reflectance profile across most of the spectrum. Region R2 is limited primarily by weaker near-IR collection, while Region R3 is limited by broader optical reflectance losses despite competitive truncated performance versus R2. Therefore, based on Question 3 evidence alone, \textbf{Region R1 is the best-performing region for spectral current generation and is the strongest candidate for tandem-relevant bottom-cell operation.}
